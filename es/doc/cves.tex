\documentclass[11pt,letterpaper,sans]{moderncv}
\moderncvtheme[black]{casual}

\usepackage[utf8]{inputenc}
\usepackage[T1]{fontenc}
\usepackage[spanish]{babel}
\usepackage{amsmath}
\usepackage{amsfonts}
\usepackage{amssymb}
\usepackage{graphicx}

\graphicspath{{../img/}}

% Personal data
\firstname{Jesid Mauricio\\}
\familyname{Mejía Castro}
\title{Ingeniero de sistemas y computación}
%\address{Cra. 13C 165 86\ }{Bogotá\ }{Colombia}
%\mobile{7}
%\phone{7}
%\email{jmauriciomejia@gmail.com}
\photo[64pt]{mauricio02.jpg}
\quote{
	\\[1cm]
	Profesional egresado de la Universidad Nacional de Colombia y estudios de posgrado (en curso) de la Universidad de Bogotá Jorge Tadeo Lozano.
	\\[1cm]
	He trabajado principalmente en el sector financiero. Tengo 4 años de experiencia trabajando con datos, incluyendo el manejo de bases de datos Oracle y SQL Server, el desarrollo de ETL con Infosphere DataStage e Integration Services y Apache Spark. Tambíen poseo habilidades de programación en los lenguajes Python y R. Apologista de los sistemas operativos basados en Unix y el software libre.
	\\[1cm]
	Me gusta conservar un activo espíritu de aprendizaje y suelo promover la excelencia técnica en el producto final. Comprometido con la honestidad y ética en los ambientes de trabajo.}

\begin{document}
\maketitle
\newpage

\section{Datos Personales} % Si deseas que tus datos se muestren como sección puedes incluirlos
\cvitem{Dirección}{Cra. 13C 165 86 -- Bogotá -- Colombia}
\cvitem{Correo}{\texttt{jmauriciomejia@gmail.com}}
\cvitem{Teléfono}{348 39 45}
\cvitem{Celular}{300 706 36 56}
\cvitem{Edad}{30}
\cvitem{Página web}{\texttt{mauriciomejia.life} (en construcción...)}

\section{Educación}
\cventry{1994--2000}{Primaria}{Centro de Asistencia Estudiantil Ronditas del Tunal}{Bogotá}{}{Cursos desde Transición hasta 4"o de primaria}
\cventry{2001--2003}{Primaria}{Gimnasio George Berkeley}{Bogotá}{}{Cursos desde 5"o de primaria hasta 7"o de bachillerato.}
\cventry{2004--2008}{Bachillerato}{Colegio Mayor José Celestino Mutis}{Bogotá}{}{Cursos desde 8"o hasta 11"o de bachillerato.}
\cventry{2009--2010}{Ingeniería de Sistemas}{Universidad El Bosque}{Bogotá}{}{Semestres 1"o, 2"o y 3"o}
\cventry{2010--2017}{Ingeniería de Sistemas y Computación}{Universidad Nacional de Colombia}{Sede Bogotá}{}{Semestres 3"o en adelante}
\cventry{2020--2022}{Maestría en Ingeniería y Analítica de Datos}{Universidad de Bogotá Jorge Tadeo Lozano}{Bogotá}{}{En curso}

\section{Experiencia Laboral}

\cventry{2016 -- 2017}
{Practicante empresarial} % Título del cargo que desempeñaste
{Enel Codensa E.S.P} % El nombre de la organización donde laboraste
{}{} % Específica el nivel de responsabilidad
{\vspace{0.05cm}
	\includegraphics[height=0.9cm]{enel.jpg} %Coloca el logo de la empresa donde laboraste
	\vspace{0.15cm}}
\cvitem{}{
	En esta empresa de servicios públicos completé mis prácticas con la Oficina Defensor del Cliente (\textit{Customer Ombudsman}). Allí se me asignó la responsabilidad de introducir nuevos ajustes a la base de datos de casos y denuncias, el cierre de indicadores del área, entre otros desarrollos \textit{in-house}, principalmente con MS Access y VBA.}
\cvitem{}
{}

\cventry{2018 -- 2021}
{Analista III y II} % Título del cargo que desempeñaste
{Banco de Bogotá S.A.} % El nombre de la organización donde laboraste
{}{}
{\vspace{0.15cm}
	\includegraphics[height=0.9cm]{bdb.png} % Coloca el logo de la empresa donde laboraste
	\vspace{0.15cm}}
\cvitem{}{
	Entre 2018 y 2019 trabajé con la División de Crédito Masivo realizando desarrollos en la base de datos (MS SQL Server 2012) y presentando reportes (principalmente en Excel) orientados a soportar la operación de estudios crediticios para clientes del banco. Entre mediados de 2019 e inicios de 2021 trabajé en la oficina de Gobierno y Calidad de la Información desarrollando ETL con DataStage y \textit{scripts} de PL/SQL con el objetivo de medir la calidad de los datos provenientes de múltiples dominios de información en la organización utilizando una bodega de datos.}
\cvitem{}
{}

\cventry{2021}
{Profesional Data Warehouse} % Título del cargo que desempeñaste
{Credivalores S.A.} % El nombre de la organización donde laboraste
{}{}
{\vspace{0.15cm}
	\includegraphics[height=0.6cm]{credivalores.jpg} %Coloca el logo de la empresa donde laboraste
	\vspace{0.3cm}}
\cvitem{}{
	Entre febrero de 2021 y mayo de 2021 trabajé en la Dirección Data Warehouse de esta entidad financiera cubriendo las necesidades de información de las áreas de producto de la compañía. Mis labores consistían en recibir el requerimiento y y desarrollarlo en todo su ciclo de vida, desde el análisis y el diseño, pasando por el desarrollo de la ETL y la construcción del modelo dimensional, hasta la visualización de los indicadores y las pruebas unitarias y de integración. Se utilizaban principalmente los productos de Microsoft: SQL Server 2017 y su lenguaje T-SQL para la base de datos, Integration Services para las ETL y Power BI para la visualización.}
\cvitem{}
{}

\newpage

\cventry{2021-2022}
{Gestor de calidad de datos e ingeniero de datos} % Título del cargo que desempeñaste
{\'{A}gata\}} % El nombre de la organización donde laboraste
{}{}
{\vspace{0.15cm}
	\includegraphics[height=1.2cm]{agata02.jpg} %Coloca el logo de la empresa donde laboraste
	\vspace{0.3cm}}
\cvitem{}{
	Desde junio de 2022 trabajo en la Agencia Anal\'{i}tica de Datos (\'{A}gata) creada como iniciativa del Distrito Capital y llamada a ser el catalizador del desarrollo de una {\em smart city}. All\'{i} desempeñé brevemente el rol de gestor de calidad de datos (2 meses). Posteriormente fui promovido al cargo de ingeniero de datos Jr. Mis labores en la Agencia consisten en preparar los datos de los diferentes proyectos que la Agencia empieza a contratar con las diferentes entidades del distrito. Entre mis actividades se incluyen el perfilamiento de los datos, su limpieza, la elaboración de los flujos de datos y ETL, el análisis dimensional (cuando se requiere), las mediciones de calidad de datos y la entrega de los datos listos para llevar a cabo los diferentes requerimientos de analítica. Todo lo anterior bajo los lineamientos de gobierno de datos que la Agencia definió. 
	Al plantearse como una organización {\em cloud-agnostic}, en la Agencia se emplean diferentes nubes, en concreto: Microsoft Azure, Google Cloud Platform y Amazon Web Services. En mi caso particular, trabajé con componentes de Azure (Azure Data Factory y Azure Databricks) y en menor medida, con algunos componentes de GCP (BigQuery, Dataflow).}
\cvitem{}
{}

\section{Habilidades en computaci\'on}
\cvcomputer{Oracle}{Modelado, DDL, DML y PL/SQL}{SQL Server}{Modelado, DDL, DML y T-SQL}
\cvcomputer{Infosphere DataStage}{Diseño e implementación de ETL}{SQL Server Integration Services}{Desarrollo de ETL}
\cvcomputer{Python}{Programación general. Análisis y tratamiento de datos con \texttt{PySpark} y \texttt{Pandas}. Visualización de datos con \texttt{matplotlib}, entre otros. paquetes}{R}{Nivel básico, estadística descriptiva, modelos de regresión lineal.}
\cvcomputer{Git}{Gestion de repositorios, experiencia con Gitflow}{GNU/Linux}{Administración general del sistema operativo, familiaridad con los comandos más utilizados (con \texttt{bash} y \texttt{zsh}).}


\section{Referencias Personales}
\textbf{Ivonne Offyr Castro Hernandez}\\
Madre\\
Cel.: 301 469 56 46\\
\\
\textbf{Rafael Alberto Mejía Cuello}\\
Padre\\
Cel.: 300 275 92 72\\
\section{Referencias Laborales}
\textbf{Cristian L. Echeverría}\\
Ing. Mecánico\\
Cel.: 300 738 22 14\\
\\
\textbf{Alfonso Arias}\\
Contador\\
Cel.: 311 466 04 51\\


\end{document}