\documentclass[11pt,letter,sans]{moderncv}
\moderncvtheme[blue]{casual}
 
\usepackage[T1]{fontenc}
\usepackage[utf8]{inputenc} %Character encoding
\usepackage[spanish,es-notilde]{babel}
%activeacute, es-lcroman, 
%es-noshorthands %Permite el uso del español

\spanishdatedel %Configuta el comando \today para que imprima <<del>> en lugar de <de>>
\usepackage{graphicx} %Permite el uso de gráficos
\usepackage[left=2cm,right=1.8cm,top=1.5cm,bottom=1.8cm]{geometry} %Establece los márgenes de la hoja
\nopagenumbers{} %Suprime la numeración de páginas, se recomienda cuando tenemos un documento pequeño
 
%Tus datos personales
\firstname{Tus nombres} 
\familyname{Tus apellidos}
%Información que se imprimirá en el margen inferior de todas las páginas 
\title{Hoja de Vida No Documentada} %Mostrará este texto debajo de tu nombre
\quote{(Última Actualización -- \today)} %Imprime la fecha de compilación
\mobile{256} %Aquí tu número de celular
\email{Aquí tu correo electrónico}
\photo[59pt]{graf/tu-foto.jpg} %Aquí tu fotografía
\address{Aquí tu dirección, la calle y el número del edificio}{Aquí la ciudad y el país si eres internacional}
%\extrainfo{Profesional en ...} 
 
\begin{document}
 
\maketitle
 
\section{Perfil}
 
\cvitem{}{Describe tu perfil y tus competencias profesionales}
 
\section{Datos Personales} % Si deseas que tus datos se muestren como sección puedes incluirlos
 
\cvitem{Estado civil}{Aquí tu estado civil ¿no es obvio?}
\cvitem{Edad}{Aquí tu edad}
\cvitem{Móvil}{Aquí tu número móvil}
\cvitem{Correo electónico}{Aquí tu correo elétrónico}
%\cvitem{Jabber ID}{Aquí tu cuenta de Jabber}
%\cvitem{Página personal}{Dirección de tu página}{<<Nombre de tu página>>}}
 
\section{Formación Académica}
 
\subsection{Profesional} % Títulos Universitarios
 
\cventry{\includegraphics[width=1.2cm]{graf/logo-de-tu-universidad.jpg} \\\centering{año}}
{Aquí tu grado académico}
{Aquí tu universidad}
{Aquí el distrito en donde está tu universidad}
{\textit{La provincia}, 
{\textbf{Aquí destaca tu mérito universitario}}}{}
 
\subsection{Especialización} % Maestrías y Diplomados
 
\cventry{\includegraphics[width=2.46cm]{graf/logo-universidad} \\ Fecha de inicio -- Fecha de finalización}
{Título de la Maestría o Diploma}
{Nombre de la Escuela donde se cursó la maestría}
{Distrito de la universidad}
{\textit{Ciudad de la univerdad}}
{\textbf{%Mérito alcanzado en la maestría o diplomado}}
 
\subsection{Complementaria}
 
\cventry{Año}{Título del curso}{Facultad}{Universidad}{\textbf{\textit{Número de horas}}}
 
\section{Conocimientos Tecnológicos}
 
\subsection{Usuario Avanzado}
 
\cvdoubleitem % Este comando permite ingresa dos ítem en una solo nivel
{\includegraphics[height=1.2cm]{graf/tulogo} \\ \textbf{Nombre del software}}{\textbf{Categoría del software}}  
{\includegraphics[height=1.2cm]{graf/tulogo} \\ \textbf{Nombre del software}}{\textbf{Categoría del software}}
 
\subsection{Usuario Intermedio}
 
\subsection{Usuario Básico}
 
\section{Idiomas}
 
\cvlanguage{Idioma}{Nivel}{Detalle del nivel}
 
\section{Experiencia Laboral}
 
\subsection{Periodo en años}
 
\cventry{Fecha de inicio -- Fecha de fin}
{Cargo} % Título del cargo que desempeñaste
{Nombre de la Empresa u Organización} % El nombre de la organización donde laboraste
{Dirección}
{Tipo de desempeño (Prácticas, Profesional)} % Específica el nivel de responsabilidad
{\vspace{0.15cm}
  \includegraphics[height=0.9cm]{graf/logo-empresa.png} %Coloca el logo de la empresa donde laboraste
  \vspace{0.15cm}}
 
\cvitem{}
{Detalle de responsabildiades} % Detalla los alcances y responsabilidades asumidas.
\cvitem{}
{\textit{Detalle de una responsabilidad menor}}
 
\end{document}